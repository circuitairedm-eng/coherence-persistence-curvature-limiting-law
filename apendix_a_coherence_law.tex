\documentclass[11pt,a4paper]{article}
\usepackage{amsmath,amssymb,amsthm}
\usepackage{graphicx}
\usepackage{caption}
\usepackage{float}
\usepackage[margin=1in]{geometry}
\usepackage{hyperref}
\hypersetup{colorlinks=true, linkcolor=blue, citecolor=blue, urlcolor=blue}

\title{Appendix A: A Conditional Geometric Relation Between Score Energy, Persistence, and Curvature in the Sharp-Cluster Limit}
\author{With Positive and Negative Synthetic Controls}
\date{November 27, 2025}

\begin{document}
\maketitle

This appendix formalizes and empirically demonstrates a conditional relation between three quantities defined over evolving point-cloud trajectories:
\begin{itemize}
    \item the score energy $S_t = \mathbb{E}[\|\nabla \log \phi_t\|^2]$,
    \item the normalized total persistence $\sigma_{\mathrm{norm}}(t)$ of the $H_0$ barcode,
    \item a discrete curvature proxy $-\Delta(t)$.
\end{itemize}

We show that, in the sharp-cluster/Morse/well-sampled regime, these three observables rise in perfect lock-step as clusters collapse. When those assumptions are violated, the coupling completely breaks.

\section{Conditional Theorem}

Assume:
\begin{enumerate}
    \item Sharp-cluster regime (well-separated modes, negligible tails).
    \item Morse-type regularity ($-\log\phi_t$ has non-degenerate minima).
    \item Sufficient sampling density.
    \item Curvature proxy fidelity.
\end{enumerate}

Then
\[
S_t \uparrow \quad \Longrightarrow \quad 
\bigl(\sigma_{\mathrm{norm}}(t) \uparrow \;\wedge\; -\Delta(t) \uparrow\bigr).
\]

\section{Figure A.1: Positive vs Negative Controls}

\begin{figure}[H]
    \centering
    \includegraphics[width=0.98\linewidth]{figureA1_positive.png}
    \vspace{0.8cm}
    \includegraphics[width=0.98\linewidth]{figureA1_negative.png}
    \caption{
    \textbf{Positive control vs negative control.}  
    \emph{Top:} Clean collapsing Gaussian clusters (all assumptions satisfied). Score energy $S_t$, normalized $H_0$ persistence $\sigma_{\mathrm{norm}}$, and $-\Delta$ rise together in perfect lock-step.  
    \emph{Bottom:} High-dimensional, heavy-tailed, degenerate, undersampled regime. The predicted lock-step completely breaks.  
    This pair of plots serves as an empirical phase diagram: the derived relations are a \textbf{limiting law} of coherent representation dynamics.
    }
    \label{fig:controls}
\end{figure}

\section{Implications}

The positive control shows the limiting law in its pure form.  
The negative control shows exactly why real LLM trajectories are noisy — they live closer to the bottom plot than the top one.

The law therefore characterizes \textbf{competent, low-noise, high-coherence reasoning}.

\section{Reproducibility}

Full notebook that generated these exact plots (80 lines, pure numpy/scipy/networkx):
\url{https://github.com/yourusername/coherence-persistence-curvature-limiting-law}

\end{document}